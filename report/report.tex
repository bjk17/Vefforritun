%% xelatex -shell-escape skil8
\documentclass[a4paper,oneside]{article}
\usepackage{a4wide}
\usepackage[icelandic]{babel}
\usepackage{fontspec}
\usepackage{xunicode}
\usepackage{graphicx}
\usepackage{enumerate}
\usepackage{hyperref}

\usepackage{minted}
\usemintedstyle{perldoc} % default, bw, perldoc
\setlength\partopsep{-\topsep} % eyðir út bili að ofan og neðan kóða
%\inputminted[ options ]{ language }{ filename }

\title{Vefforritun}
\date{\today}
\author{Bjarni Jens Kristinsson \\ Guðrún Þóra Sigurðardóttir \\ Kristrún Skúladóttir}

\begin{document}
\maketitle

\section{Hugmynd}
Hugmyndin var að sækja gögn af \href{http://docs.apis.is/}{APIs} t.d.\@ um bíómyndir
og birta þær á ákveðinn hátt og geyma gögn í gagnagrunni.

Okkur datt í hug að útbúa vefsíðu þar sem notandi gæti skoðað hvaða myndir verið er
að sýna í bíó og valið úr þeim þær myndir sem hann langaði að sjá. Eins konar ,,to see``
lista, eða eins og við kjósum að kalla það ,,Myndir sem mig langar ógeðslega mikið að sjá``.

\section{Framkvæmd}
Við völdum að nota Python Django bakenda og JavaScript/jQuery fyrir asynchronous aðgerðir.

Við gerðum MoSCoW greiningu, þ.e.\@ við áætluðum hvaða virkni væri nauðsynleg, minna nauðsynleg
og hverju væri skemmtilegt að bæta við. Einnig áætluðum við hvaða virkni við vildum ekki hafa. \\

\noindent\textbf{MUST}
\begin{itemize} 
    \item Sækja bíómyndir af apis.is.
    \item Birta bíómyndirnar.
    \item Birta bíó og sýningartíma fyrir hverja mynd.
    \item Gefa hverjum notanda sitt einstaka id.
    \item Virkni til að leyfa notanda að velja þær myndir sem hann vill sjá.
    \item Vista valdar myndir í gagnagrunni fyrir hvern notanda.
    \item Birta lista yfir valdar myndir þegar notandinn fer aftur inn á slóðina með sínu id.
\end{itemize}

\noindent\textbf{SHOULD}
\begin{itemize} 
    \item Birta valmynd með bíóhúsum.
    \item Cache-a niðurstöður frá apis í ákveðinn tíma.
\end{itemize}

\noindent\textbf{COULD}
\begin{itemize}
    \item Innskráning með lykilorði.
    \item Sýna myndir sem búið er að sjá á ákveðnum stað.
\end{itemize}

\noindent\textbf{WON'T}
\begin{itemize}
    \item Ble
\end{itemize}

\section{Niðurstaða}
Lýsa hverju við náðum að afreka?

\end{document}