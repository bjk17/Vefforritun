%% xelatex -shell-escape skil8
\documentclass[a4paper,oneside]{article}
\usepackage{a4wide}
\usepackage[icelandic]{babel}
\usepackage{fontspec}
\usepackage{xunicode}
\usepackage{graphicx}
\usepackage{enumerate}
\usepackage{hyperref}

\usepackage{minted}
\usemintedstyle{perldoc} % default, bw, perldoc
\setlength\partopsep{-\topsep} % eyðir út bili að ofan og neðan kóða
%\inputminted[ options ]{ language }{ filename }

\title{Vefforritun}
\date{\today}
\author{Bjarni Jens Kristinsson \\ Guðrún Þóra Sigurðardóttir \\ Kristrún Skúladóttir}

\begin{document}
\maketitle

\section{Hugmynd}
Hugmyndin var að sækja gögn af \href{http://docs.apis.is/}{APIs} t.d.\@ um bíómyndir
og birta þær á ákveðinn hátt og geyma gögn í gagnagrunni.

Okkur datt í hug að útbúa vefsíðu þar sem notandi gæti skoðað hvaða myndir verið er
að sýna í bíó og valið úr þeim þær myndir sem hann langaði að sjá. Eins konar ,,to see’’
lista eða ,,Mínar myndir’’.

\section{Framkvæmd}
Við völdum að nota Python Django bakenda og JavaScript/jQuery fyrir asynchronous aðgerðir.
Ástæðan fyrir því að við völdum Django var að það er algengur rammi (e.\@ framework) auk þess
sem okkur langaði að auka kunnáttu okkar í Python.

Við gerðum MoSCoW greiningu, þ.e.\@ við áætluðum hvaða virkni væri nauðsynleg, gagnleg og
hvað væri gaman að hafa aukalega. Einnig áætluðum við hvaða virkni við vildum ekki hafa. \\

\noindent\textbf{MUST}
\begin{itemize} 
    \item Sækja bíómyndir af apis.is.
    \item Birta bíómyndirnar.
    \item Birta bíó og sýningartíma fyrir hverja mynd.
    \item Gefa hverjum notanda sitt einstaka id.
    \item Virkni til að leyfa notanda að velja þær myndir sem hann vill sjá.
    \item Vista valdar myndir í gagnagrunni fyrir hvern notanda.
    \item Birta lista yfir valdar myndir þegar notandinn fer aftur inn á slóðina með sínu id.
\end{itemize}

\noindent\textbf{SHOULD}
\begin{itemize} 
    \item Birta valmynd með bíóhúsum.
    \item Cache-a niðurstöður frá apis í ákveðinn tíma.
\end{itemize}

\noindent\textbf{COULD}
\begin{itemize}
    \item Innskráning með lykilorði.
    \item Sýna myndir sem notandi hefur séð á ákveðnum stað.
    \item Sía út myndir ef notandi er of ungur.
\end{itemize}

\noindent\textbf{WON'T}
\begin{itemize}
    \item Kommentakerfi.
    \item Notendur geta tengst vinum, og séð þær myndir sem báðir aðilar viljas sjá.
        Eins konar ,,social network’’ fyrir bíó.
\end{itemize}

\section{Niðurstaða}
Við teljum okkur hafa verið nokkuð raunsæ þegar við gerðum MoSCoW greininguna. 
Við náðum að klára allt í \emph{MUST} og \emph{SHOULD}, en þó ekki meira en það. 

Verkefnin sem við þurftum að leysa komu í törnum, um leið og við leystum
eitt vandamál komu í ljós fleiri verkefni sem þyrfti að leysa.
Sem dæmi má taka þurfti oftar en ekki að bæta við eða lagfæra viðeigandi JavaScript fall.

Helstu vandræðin sem við lentum í voru með JQuery og vafra í snjallsímum, þar sem
\texttt{on click} virkaði ekki alveg eins og við héldum. Auk þess hættu Apis að 
veita aðgang að þessum gögnum þremur dögum fyrir skil, sem gæti talist nokkuð óheppilegt
í okkar sporum. 

Það sem okkur finnst að hefði mátt betur fara væri helst uppsetning gagnagrunns. Eftirá að hyggja hefði skipulag gagnagrunnsins geta verið öðruvísi til að henta okkar verkefni betur. Það hefði getað einfaldað forritskóða og jafnvel getað fækkað köllum í gagnagrunninn. Þar með hefði sá tími sem tekur síðuna að hlaðast getað minnkað. Við lærðum þó mikið af þessu þannig að næst myndum við eyða meiri tíma í að rissa upp beinagrindina af uppsetningu gagnagrunns og kóðans.

Þegar á heildina er litið erum við mjög stolt af því sem við náðum að afreka. Það var skemmtilegt og lærdómsríkt að kljást við hagnýtt verkefni. 

\end{document}